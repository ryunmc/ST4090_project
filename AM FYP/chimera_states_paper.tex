\documentclass{article}
\usepackage{amsmath}
\usepackage{amsfonts}
\usepackage{amsfonts}
\usepackage{amssymb}
\usepackage{graphicx}
\usepackage{float}
\usepackage{pifont}
\usepackage{pdfpages}
\usepackage{pgfplots}
\usepackage{tikz}

\newenvironment{definition}[1][Definition]{\begin{trivlist}
\item[\hskip \labelsep {\bfseries #1}]}{\end{trivlist}}

\newtheorem{thm}{Theorem}[section]
\newtheorem{dfn}{Definition}[section]
\newtheorem{pf}{Proof}[section]
\newtheorem{lm}{Lemma}[section]
\begin{document}
\title{Chimera States}
\author{Ryan McCarthy}
\date{} %makes the date blank
\maketitle

\section{Introduction}
Dynamical systems can sometimes be in a state of synchronisation or chaos. A `chimera state" is when both chaos and order is present in a dynamical system.

\section{The Model}

\label{sec:The ODE}

The model consists of a ring of $N$ phase oscillators or nodes. Coupling is introduced between particular oscillators, and the coupling distribution affects the resulting dynamics. The phase $\phi_i$ of the $i^{th}$ oscillator is governed by the following ODE:

\begin{equation}
\label{eq:ODE}
\frac{\partial \phi_i}{\partial t} = \omega - \frac{1}{n_i} \sum\limits_{j\in{\mathcal{N}_i}}^{} G(i,j)\sin(\phi_i - \phi_j + \alpha), \hspace{5mm}i = 0\ldots N-1 
\end{equation}
\\
$\mathcal{N}_i$ is a set consisting of nodes connected to node $i$, the cardinality of this set being $n_i$. $\mathcal{N}_i$ will be referred to as node $i's$ "neighbours". $\omega$ is set to zero (as suggested in \cite{bib:AbramsStrogatz}) and $0\leq\alpha \leq\frac{\pi}{2} $ is an angle. $G(i,j)$ is the strength of the coupling between nodes $i$ and $j$, the higher the magnitude, the higher the expected synchronicity and vice-versa.
\\
\\
To avoid ambiguity in notation from herein $i$ refers to the imaginary unit and $m$ and $n$ to refer to the $m^{th}$ and $n^{th}$ oscillators. 

\begin{dfn}[All-to-all coupling]
If node $m$ is coupled to all other nodes in the network, then $G(m,n)=1, \hspace{5mm}n = 0\ldots N-1$
\end{dfn}

\begin{dfn}[Nearest $k$ neighbours]
If node $m$ is coupled to its nearest $k$ neighbours, then:

$$G(m,n) = \begin{cases} 
      1 & |m-n|\leq k \\
      0 & |m-n| > k 
   \end{cases} $$
\end{dfn}
\noindent
Let $\Omega$ denote the angular frequency of a rotating frame in which the dynamics simplify as much as possible, and let
$$\theta = \phi - \Omega t$$
denote the phase of an oscillator relative to this frame.
\\
\\
A complex order parameter of the form $R_me^{i\Theta m}$ is introduced which measures how synchronised oscillator $m$ is with respect to its neighbours $\mathcal{N}_m$ (see \ref{eq:ODE}). Thus, the magnitude of the order parameter is $R_m$. The order parameter is defined as follows:

\begin{equation}
R(x)e^{i\Theta(x)} = \int_{-\pi}^{\pi}G(x-x')e^{i\theta(x')}dx'
\end{equation}
which in discrete form, becomes:

\begin{equation}
\label{eq:order_param}
R_me^{i\Theta_m} = \frac{1}{n_m}\sum\limits_{n\in{\mathcal{N}_m}}^{} G(m,n) e^{i\theta_n}
\end{equation}

$R_m$ will evaluate to a number between $0$ and $1$, indicating no synchronisation and full synchronisation respectively. Note that throughout $\theta$ is equivalent to $\theta_m$. Next, a useful ODE for $\theta$ is derived.
\\
\\
\textbf{Claim}
\begin{equation}
\label{eq:thing1}
\frac{\partial \theta_m}{\partial t} = \omega - \Omega - R_m \sin[\theta_m - \Theta_m + \alpha]
\end{equation}
\\
\\
\textbf{Proof}

$$\theta = \phi - \Omega t \implies \dot{\theta} = \dot{\phi} - \Omega$$
\begin{equation}
\label{eq:thing2}
\dot{\theta} = \omega - \Omega - \frac{1}{n_m} \sum\limits_{n\in{\mathcal{N}_i}}^{} G(m,n)\sin[\phi_m - \phi_n + \alpha]
\end{equation}
comparing \ref{eq:thing2} to \ref{eq:thing1}, we see that what we are trying to show is equivalent to:
$$R_m \sin[\theta_m - \Theta_m + \alpha] = \frac{1}{n_m} \sum\limits_{n\in{\mathcal{N}_i}}^{} G(m,n)\sin[\phi_m - \phi_n + \alpha]$$
using the equation defining the order parameter:

$$R_m e^{i(\Theta_m - \theta_m - \alpha)} = \frac{1}{n_m} \sum\limits_{n\in{\mathcal{N}_i}}^{} G(m,n) e^{i(\theta_n - \theta_m - \alpha)}  $$

equate the imaginary part of both sides:

$$R_m \sin[\Theta_m - \theta_m -\alpha] = \frac{1}{n_m} \sum\limits_{n\in{\mathcal{N}_i}}^{} G(m,n) \sin[\theta_n - \theta_m - \alpha]  $$


$$\text{recall,} \; \theta = \phi - \Omega t $$
$$\implies \theta_m - \theta_n = \phi_m - \phi_n$$
$$ \text{hence} \; R_m \sin[\theta_m - \Theta_m + \alpha] = \frac{1}{n_m} \sum\limits_{n\in{\mathcal{N}_i}}^{} G(m,n)\sin[\phi_m - \phi_n + \alpha]$$

this proves the required result.

\begin{thm}[Self-consistency equation for oscillators]
This is given by:

$$R_m\exp[i\Theta_m] = e^{i\beta}\frac{1}{n_m} \sum\limits_{n\in{\mathcal{N}_m}}^{} G(m,n)\exp[i\Theta_n] \times \frac{\Delta + \sqrt{ \Delta^2 - {R_n}^2}}{R_n}  $$
where $\beta = \frac{\pi}{2}-\alpha$ and $\Delta = \omega - \Omega$

\end{thm}

\begin{pf}
Recall for oscillator $m$:

$$\frac{\partial \theta_m}{\partial t} = \omega - \Omega - R_m \sin[\theta_m - \Theta_m + \alpha]$$
\noindent
the resulting dynamics will depend on the magnitude of $R_m$ relative to $|\omega - \Omega|$. The first case is when $R_m \geq |\omega - \Omega|$, the second is when $R_m < |\omega - \Omega|$. Note $\frac{\partial \theta_m}{\partial t}$ and $\dot{\theta}$ will be used interchangeably.

\textbf{Case one $R_m \geq |\omega - \Omega|$}

\vspace{1 cm}
\begin{figure}[H]
\begin{tikzpicture}{}
\begin{axis}[
    axis x line=middle,
    axis y line=middle,
    axis on top,
    tick label style={font=\tiny},
    smooth,
    xlabel=$\theta$,
    ylabel=$\dot{\theta}$,
    xmin=0, xmax=500,
    ymin=-5, ymax=5,
    width=0.6\textwidth,
    height=0.6\textwidth,
    legend style={at={(0.02,0.97)},anchor=north west},
]
\addplot[draw=blue,domain=0:350]{2-3*sin(x)};
\end{axis}
\end{tikzpicture}
\end{figure}
it can be seen that there are two equilibria in figure. Let $\theta^1$ denote the stable root and $\theta^2$ denote the unstable root. In this case $\dot{\theta}$ will have a root where

$$ \frac{\omega - \Omega}{R(x)} = \sin[\theta(x) - \Theta(x) + \alpha] $$

using the identity $\sin^2(A) + \cos^2(A) = 1$ yields:

$$\cos^2[\theta(x) - \Theta(x) + \alpha] = 1- \displaystyle(\frac{\omega - \Omega}{R(x)})^2 $$
the right hand side can be rearranged to yield:

$$\cos^2[\theta(x) - \Theta(x) + \alpha] =  \frac{R(x)^2-(\omega-\Omega)^2}{R(x)^2}$$

this yields:

$$\cos[\theta(x) - \Theta(x) + \alpha] =  \pm \sqrt{\frac{R(x)^2-(\omega-\Omega)^2}{R(x)^2}} $$

to ascertain which root corresponds to $\theta^1$ and which corresponds to $\theta^2$, let

$$f(\theta) := \omega - \Omega - R_m \sin[\theta - \Theta + \alpha]$$.

Recall our stable equilibrium must satisfy:

$$ \frac{df}{d\theta}\restriction_{\theta = \theta^1} < 0$$

however,

$$ \frac{df}{d\theta_m} = -R_m\cos[\theta(x) - \Theta(x) + \alpha] $$

which tells us the stable root $\theta_s$ must satisfy:

$$\cos[\theta(x) - \Theta(x) + \alpha] =  \sqrt{\frac{R(x)^2-(\omega-\Omega)^2}{R(x)^2}} $$

hence:

$$e^{i(\theta(x)-\Theta(x) +\alpha)} = \cos[\theta(x)-\Theta(x) +\alpha] + i\sin[\theta(x)-\Theta(x) +\alpha]$$

$$=\frac{\sqrt{R(x)^2-\Delta^2} + i\Delta}{R(x)} $$
$$=i \frac{\sqrt{\Delta^2-R(x)^2}+\Delta}{R(x)}$$

recall the definition of the order parameter: 

$$R_me^{i\Theta_m} = \frac{1}{n_m}\sum\limits_{n\in{\mathcal{N}_m}}^{} G(m,n) e^{i\theta_n}$$
however:

$$e^{i\theta_n} = e^{i(\theta_n - \Theta_n + \alpha)}e^{-i\alpha}e^{i\Theta_n}$$

filling in to the above yields:

$$R_m e^{i \Theta_m} = ie^{-i\alpha}\sum G(m,n)e^{i\Theta_n} \times \frac{\sqrt{\Delta^2-R(x)^2}+\Delta}{R(x)} $$

Now let $\beta = \frac{\pi}{2} - \alpha$, this is equivalent to:

$$R_m e^{i \Theta_m} = ie^{-i\alpha}\sum_{n \in \mathcal{N}_m} G(m,n)e^{i\Theta_n} \times \frac{\sqrt{\Delta^2-R(x)^2}+\Delta}{R(x)} $$

\textbf{Case two $R_m < |\omega - \Omega|$}


\begin{tikzpicture}
\begin{axis}[
    axis x line=middle,
    axis y line=middle,
    axis on top,
    tick label style={font=\tiny},
    smooth,
    xlabel=$\theta$,
    ylabel=$\dot{\theta}$,
    xmin=0, xmax=350,
    ymin=-5, ymax=5,
    width=0.6\textwidth,
    height=0.6\textwidth,
    legend style={at={(0.02,0.97)},anchor=north west},
]
\addplot[draw=blue,domain=0:500]{3-2*sin(x)};
\end{axis}
\end{tikzpicture}

In this case we see that there are no stationary solutions of $\dot{\theta}$. However, can perform some analysis. Let $p(\theta(x))$ denote the probability that 

\section{Self-consistency equation for $k$ nearest neighbours and fully connected cases}

\textbf{Fully connected nodes} \; Let $N$ denote the number of nodes in the network and suppose node $m$ is fully connected. We then have:

\begin{equation}
R_me^{i\Theta_m} = \frac{1}{N}\sum\limits_{n=1}^{N} e^{i\theta_n}
\end{equation}

as before:

$$e^{i(\theta_n-\Theta_n +\alpha)} = i \frac{\sqrt{\Delta^2-R_n^2}+\Delta}{R_n}$$
using $e^{i\theta_n}= e^{i(\theta_n-\Theta_n+\alpha)e^{-i\alpha}}e^{i\Theta_n}$ yields:

$$R_me^{i\Theta_m} = e^{i\beta} \frac{1}{N}\sum\limits_{n=1}^{N} e^{i\Theta_n}\times \frac{\sqrt{\Delta^2-R_n^2}+\Delta}{R_n}$$

Note that $\Theta$ comes from the value of the order parameter, and $\theta$ is the relative phase of an oscillator

\textbf{Nearest k neighbours} \; Under the nearest k neighbours coupling, the order parameter fulfills:

\begin{equation}
R(x)e^{i\Theta(x)} = \int_{x-k}^{x+k}e^{i\theta(x')}dx'
\end{equation}

\end{pf}



\end{document}